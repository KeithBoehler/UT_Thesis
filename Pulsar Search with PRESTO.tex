\documentclass{article}
\usepackage{hyperref}
\usepackage{listings}

\title{LoFASM Pulsar Search with PRESTO}
\author{Keith E. Boehler Jr.}
\date{}
\begin{document}
   \maketitle
   Hello world!
   
   \section{Installing Presto}
   The instructions here are based on those by Scott Ransom's Install page on github as well as other 
   tips found on the internet. The following will be centered on using Ubuntu 20.04. Tips for macOS Mojave
   will be provided as well. 
   
   \subsection{FFTW 3}
    This one is rather straight forward. 
    
    
    \begin{enumerate}
    	\item At the time of this writing the latest version is 3.3.8 and can be downloaded \href{http://www.fftw.org/download.html}{from this link}
    	\item \noindent Once the source has been downloaded it can be unpacked using the command:
	    \begin{lstlisting}[language=bash]
	    $ tar -zx fftw-x.x.x.tar.gz 
	    \end{lstlisting}
    	(Will need to replace the x.x.x with the current version number). And enter the unpacked directory.
    	\item \noindent After that we need to configure. According to Scott there are some suggested flags if you are
    	using an Intel chip, which I am. So my command will look like: 
	    \begin{lstlisting}[language=bash]
	    ./configure --prefix=$HOME \
	    --enable-shared --enable-single \
	    --enable-sse --enable-sse2 --enable-avx \
	    --enable-avx2 --enable-fma
	    \end{lstlisting}
	    \item \noindent Followed by: \begin{lstlisting}[language=bash]
	    $ make
	    \end{lstlisting}
	    \item \noindent Followed by: 
		\begin{lstlisting}[language=bash]
		$ make check
		\end{lstlisting}
	    \item \noindent Followed by: 
		\begin{lstlisting}[language=bash]
		$ make install
		\end{lstlisting}
	    \item \noindent Followed by:
		\begin{lstlisting}[language=bash]
		 $ make installcheck
		\end{lstlisting}
    	\end{enumerate}
    	Due to the fact that we used \$HOME in our prefix we should have /bin /include and other directories
    	being made in our home directory. This will help later when we are making our environment variables
    	as the location of configure files and executable will be in more standard location.
    
    	\subsection{PGPLOT}
    	This one is a bit involved, so I recommend brewing your favorite cup of Joe before getting started. This
    	portion is based on the PGPLOT’s instructions for all UNIX systems. 
    	
    	\begin{enumerate}
    	\item \noindent I downloaded my source for PGPlot from \href{ftp://ftp.astro.caltech.edu/pub/pgplot/pgplot5.2.tar.gz}{here, version 5.2.}. And untarred with: 
    	\begin{lstlisting}[language=bash]
    	$ tar -xzvf pgplot5.2.tar.gz
    	\end{lstlisting}
    	 At this point it is prudent to create what the pgplot unix instructions call the target directory, that is where we will install. I find it to be clear to rename the directory we just unpacked to \texttt{pgplot\_src} and the actual install directory to be \texttt{pgplot}. 
    	\item Copy the drivers list from the source directory (\texttt{pgplot\_src}) into our install directory (\texttt{pgplot}). If you are inside the install directory here is a snippet: \noindent
    	\begin{lstlisting}[language=bash]
    	$ cp ../pgplot_src/drivers.list .
    	\end{lstlisting} 
    	\item 
    	\end{enumerate}
    	


    	
\end{document} 
